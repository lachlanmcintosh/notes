\section{Algebra: Rings, Modules and Fields}

\subsection{QUESTIONS}
\begin{compactitem}
\item In our definition of a field we do not explitly state that it must be unital, though we do in an integral domain. Is the requirement of unital implicit or not nessecary?
\item 
\end{compactitem}

\subsection{Background}
\begin{T} The division algorithm for \e{$\mathbb{N}$}. Take \e{$b\leq a$}. Then \e{$\exists q,r\in \mathbb{Z}$} s.t. \e{$a=qb+r, 0\leq r <b$}. Moreover \e{$q$} and \e{$r$} are unique. \end{T}
\begin{PP} not here yet \end{PP}

\begin{T} The euclidean algorithm uses the division algorihtm: iterate using \e{$r_{k-2}=qr_{k-1}+r_k$}. The gcd is the last number you write down \end{T}

\begin{D}
A \dw{group} is a set, \e{$G$}, together with an operation \e{$\cdot$} that combines any two elements \e{$a$} and \e{$b$} to form another element, denoted \e{$a\cdot b$} or \e{$ab$}. To qualify as a group, the set and operation, \e{$(G,\cdot )$}, must satisfy four requirements known as group axioms. \begin{compactitem}
\item[\dw{Closure}] \e{$\forall a,b\in G, a\cdot b \in G$}
\item[\dw{Associativity}] \e{$\forall a,b,c \in G, (a\cdot b)\cdot c = a\cdot (b\cdot c)$}
\item[\dw{Identity element}] \e{$\exists e \in G$} s.t. \e{$\forall a \in G, e \cdot a = a\cdot e=a$}. Such an element is unique.
\item[\dw{Inverse element}] \e{$\forall a \in G , \exists b \in G$} s.t. \e{$ a\cdot b = b \cdot a = e$}
\end{compactitem}
\end{D}
\begin{R} A group is so icky: SO ICIA: a set and an operation with the identity, closure, inverse, and associativity requirements.\end{R}

\begin{D} Groups for which the \dw{commutativity equation} \e{$a \cdot b = b \cdot a$} always holds are called \dw{abelian groups} \end{D}
\begin{R} An abelian group is so icky and yuck: SO ICIAC. \end{R}

\begin{D}
A \dw{ring} is a (nonempty) set \e{$R$} with the two binary operations \e{$+,\times$} satisfying the following conditions 
\begin{compactitem}
\item \e{$(R,+)$} forms an abelian group
\item The operation \e{$\times$} is associative: that is \e{$\forall x,y,z\in R, x\times (y \times z) = (x \times y) \times z$}
\item The left and right \dw{distriutive laws} hold: that is \e{$\forall x,y,z \in R, x\times (y+ z) = (x\times y)+ (x\times z), (y+z)\times x = (y\times x)+(z\times x)$}
\end{compactitem} A ring will be denoted \e{$(R,+,\times)$} \end{D}
\begin{R} A ring is SOO ICIAC, and has ADD \end{R}

\begin{D} If \e{$\exists e \in R$} s.t. \e{$\forall a\in R, ea=a=ae$}, we call the element \e{$e$} the \dw{multiplicative identity}. It is unique and is usually denoted by \e{$1$}. \end{D}

\begin{D} A ring \e{$(R,+,\times)$} is said to be \dw{commutative} if multiplication is commutative. \end{D}
\begin{D} A ring \e{$(R,+,\times)$} is said to be \dw{unital} if there exists an identity for multiplication (\e{$1$}) and it is not also \e{$0$}. \end{D}
\begin{R} Unital describes sets. If unital unital describes sets, unit describes elements. a unit set akes no sense, a unital set sounds right. \end{R}
\begin{R} The unital set contians one element?\end{R}

\begin{D} Let \e{$X$} be any set and denote by \e{$\mathcal{P}(X)$} the \dw{power set} of \e{$X$} to be the set of all possible subsets. Elements of the power sets are sets.
Operations on this set are 
\begin{compactitem}
\item \e{$A+B=(A\cup B)\backslash(A\cap B)$}
\item \e{$A\times B=A\cap B$}
\end{compactitem}
Then \e{$(\mathcal{P}(X),+,\times)$} s a commutative unital ring \end{D}
\begin{R} The power set makes sets out of elements \end{R}

\begin{D} Let \e{$R$} be a unital ring. An element \e{$a\in R$} is a \dw{unit} if there exists \e{$b\in R$} s.t. \e{$ab=ba=1$}. The element \e{$b$} is called the \dw{multiplicative inverse} of \e{$a$} and is denoted \e{$a^{-1}$}.
\end{D}
\begin{R} Every element that has a multiplicative inverse is a unit.  \end{R}
\begin{D} The ring \e{$R$} is called a \dw{ division ring} if every non-zero element is a unit. \end{D}
\begin{R} A division ring is simply a ring in which division can occur with every element. Also \e{$\forall d\in D\backslash\{0\}, d|1$} \end{R}
\begin{D} A \dw{field} is a commutative division ring. An alternative definition is:
\e{$(F,+,\times)$} is a field iff:\begin{compactitem}
\item the algebraic structure \e{$(F,+)$} is an abelian group
\item the algebraic structure \e{$(F\backslash\{0\},\times)$} is an abelian group
\item the operation \e{$\times$} distributes over \e{$+$}.
\end{compactitem}
\end{D}
\begin{R} A field is a commy cutting rings \end{R}
\begin{D} If \e{$a,b \in R$} are non zero elements in a ring \e{$R$} satisfying \e{$ab=0$} then they are called \dw{zero-divisors} \end{D}
\begin{R} \e{$a$} is a zero divisor iff it divides zero: \e{$a|0$}. \end{R}
\begin{Le}[Cancellation Law] Let \e{$R$} be a ring. Then \e{$R$} has no zero-divisors iff the following condition holds \e{$\forall a,b,c \in R$} with \e{$a\not= 0$}.
$$\e{ab=ac \implies b=c}$$
$$\e{ba=ca \implies b=c}$$
\end{Le}
\begin{R} The cancelation law applies if there are no zero divisors. Require the cancelled element to be non-zero. NODOZ: NZDZ: No zero divisors or Zero \end{R}

\begin{D} An \dw{integral domain} is a commutative unital ring in which there are no zero diviors. \end{D}
\begin{R} integral domains are curzd \end{R}

\begin{T} Every field is an integral domain \end{T}
\begin{R} Fields have less letters and thus less elements than integral domains. \end{R}

\begin{T} Every finite integral domain is a field. \end{T}
\begin{R} A FInitE integraL Domain = a FIELD. Show using self map. \end{R}









\subsection{Subrings and Ideals}
\begin{D} A \dw{subring} of a ring $R$ is a subset \e{$S\subset R$} which, when equipped with the operations from \e{$R$} forms a ring. \end{D}
\begin{R} A subring has sub for subset and ring for ring operations \end{R}

\begin{Le} A non-empty subset \e{$S\subset R$} of a ring \e{$(R,+,\times)$} is a subring iff it satisfies the following for all \e{$a,b\in S$}: \begin{compactitem}
\item \e{$a-b\in S$}
\item \e{$a\times b \in S$}
\end{compactitem}
\end{Le}
\begin{R} A child of a ring becomes a ring when it becomes closed. Open rings are adolescents. \end{R}

\begin{D} An \dw{Ideal} of a ring \e{$R$} is a non-empty subset \e{$I\subseteq R$} that satisfies the following for all \e{$a,b \in I$} and \e{$r\in R$}.
\begin{compactitem}
\item \e{$a-b\in I$}
\item \e{$a\times r \in I, r \times a \in I$}
\end{compactitem}
We deonte this by \e{$I\triangleleft R$}
\end{D}
\begin{R} It is ideal when a child can deal with what a parent can. \end{R}

%\begin{Le} \e{$I\triangleleft R, I=R \implies u \in I, \text{where } \exists v, uv=1$} \end{Le}
\begin{Le} A commutative unital ring \e{$R$} has exactly two ideals \e{$\Longleftrightarrow $} \e{$R$} is a field. \end{Le}
\begin{R} A field is always CURTIous. CURTIous rings are always fields. Thus Commy cutting rings are both CURZD and CURTIous \end{R}








\subsection{Homomorphisms and Quotients}
\begin{D} A function \e{$f: A \to B$} is \dw{injective} (one-to-one) if every element of the codomain is mapped to by at most one element of the domain. \e{$\forall x,y \in A, x\not= y \implies f(x)\not= f(y)$}. An injective function maps distinct objects to distinct objects.
\end{D}
\begin{R} Think how hard it is for a needle to hit the same sot twice. Injection is injective.  \end{R}

\begin{D} A function is \dw{surjective} (onto) if every element in the codomain is mapped to by at least one element of the domain. That is; the image and the codomain are equal. \e{$\forall y \in B, \exists x\in A\ s.t.\ y=f(x)$} \end{D}
\begin{R} A surjective function is one the fills the codomain.  \end{R}
\begin{D} A \dw{bijective funciton} is both surjective and injective. \end{D}
\begin{T} For every function \e{$f$}, subset \e{$A$} of the domain and subset \e{$B$} of the codomain we have \e{$A \subset f^{-1}(f(A))$} and \e{$f(f^{-1}(B)) \subset B$}. If \e{$f$} is injective we have \e{$A = f^{-1}(f(A))$} and if \e{$f$} is surjective we have \e{$f(f ^{-1}(B)) = B$}.
\end{T}

\begin{D} A \dw{ring homomorphism} is a map \e{$\varphi:R\mapsto S$} between rings s.t. \e{$\forall a,b \in R$} 
\begin{compactitem}
\item \e{$\varphi (a+b)=\varphi (a)+\varphi (b)$} (note this is the condition that \e{$\varphi$} is a homomorphism of the underlying abelian group for addition)
\\item \e{$\varphi (ba)=\varphi (a)\varphi (b)$} (this is a condition that \e{$\varphi$} is a homomorphism of the underlying abelian group of multiplication)
\end{compactitem}
\end{D}

\begin{D} A \dw{isomorphism} of two rings is a bijective homomorphism \end{D}
\begin{D} If there is a isomorphism between two rings, they are said to be \dw{isomorphic} \end{D}
\begin{R} Isomorphic rings morph into each other \end{R}

\begin{Le} Let \e{$\varphi : R \to S$} be a homomorphism. \begin{compactitem}
\item The \dw{kernal} of \e{$\varphi, ker(\varphi ) = \{r\in R: \varphi(r)=0\}$} is an ideal in \e{$R$}.
\begin{R} The kernal is does not interact with strangers, thus the kernal is the ideal child. \end{R}
\item The \dw{image} (the output of \e{$\varphi$}) of \e{$\varphi$} is a subring of \e{$S$}.
\end{compactitem}
\end{Le}
%\begin{PP} see page 8 \end{PP}

\begin{Le} A homomorphism \e{$\varphi$} is injective iff \e{$ker(\varphi )=0$}. \end{Le}
\begin{R} An injective function is an injection. People who do injections are 0-prospect people. \end{R}

\begin{D} Let \e{$I\triangleleft R$} be an ideal in a ring \e{$R$}. Denote by \e{$R\backslash I$} (called the \dw{quotient ring}) to be the set of (additive) cosets of \e{$I$} in \e{$R$}. 
$$\e{R/I = \{a+I:a\in R\}}$$ 
We define operations on this set by 
$$\e{(a+I)+(b+I)=(a+b)+I}$$
$$\e{(a+I)\times(b+I) = ab+I}$$
\end{D}
\begin{R} The quotient ring is the ring in which elements of \e{$I$} are ignored and have been given the value of zero. \end{R}

\begin{R} The \dw{equivalence class} of the element \e{$a$} in \e{$R$} is given by
$$\e{[a] = a + I := \{ a + r : r \in I \}}$$
This equivalence class is also sometimes written as \e{$a mod I$} and called the \dw{residue class of a modulo I}.
\end{R}

\begin{R} \e{$\mathbb{R}[X]/(X^2+1)\cong \mathbb{C}$} since we force \e{$X^2 + 1$} to be zero \end{R}

\begin{R} If \e{$R$} is commuttaive then so too is \e{$R/I$}. The converse is not true in general. \end{R}

\begin{R} The \dw{Chinese remainder theorem} states that, if the ideal \e{$I$} is the intersection (or equivalently, the product) of pairwise coprime ideals \e{$I_1,\cdots,I_k$}, then the quotient ring \e{$R/I$} is isomorphic to the product of the quotient rings \e{$R/I_p, p=1,\cdots,k$}. \end{R}

\begin{Le} The \dw{finite cyclic group} of order \e{$n$} is isomorphic to \e{$\mathbb{X}/n\mathbb{Z}$} \end{Le}

\begin{R} The \dw{zero element} is the element \e{$0+I$} and the multiplicative identity is \e{$1+I$}. \end{R}

\begin{D} If \e{$G$} is a group, and \e{$H$} is a subgroup of \e{$G$}, and \e{$g\in G$} then:
\begin{compactitem}
\item \e{$gH = \{gh : h \in H\}$} is a \dw{left coset} of \e{$H$} in \e{$G$}, and
\item \e{$Hg = \{hg : h \in H\}$} is a \dw{right coset} of \e{$H$} in \e{$G$}.
\end{compactitem}
Only when H is \dw{normal} will the right and left cosets of H coincide, which is one definition of normality of a subgroup.
\end{D}

%See the example here for quotient groups http://en.wikipedia.org/wiki/Quotient_group. 


\begin{Le} Let \e{$R$} be a ring. Given an ideal \e{$I\triangleleft R$}, the \dw{natural projection} map 
$$\e{\varphi : R\to R/I, \varphi(a)=a+I}$$
is a surjective ring homomorphism with \e{$ker(\varphi )= I$}. \end{Le}

%\begin{PP} see page 10 \end{PP}
\begin{R} As a general rule, the smaller the ideal the larger the quotient ring. \end{R}


\subsubsection{Isomorphism theorems}
\begin{T} Let \e{$\varphi: R\to S$} be a ring homomorphism. Then 
$$\e{R/ker(\varphi ) \cong Im(\varphi)}$$
That is; the image of \e{$\varphi$} is isomorphic to the quotient group \e{$R/ker(\varphi)$}.
\end{T}
\begin{T} Let \e{$R$} be a ring.
Suppose \e{$I\triangleleft R$} is an idea and \e{$S\leq R$} is a subring. Then
$$\e{(S+I)/I\cong S/(S\cap I)}$$
\end{T}
\begin{T} Suppose that \e{$I,J \triangleleft R$} are ideals in \e{$R$}, and \e{$I\subseteq J$}. Then 
$$\e{ (R/I)/(J/I)\cong R/J}$$
\end{T}
Here it is unerstood that part of the assertion being made is that each expression makes sence within these statements.

\begin{Le} Let \e{$\varphi : R \to R'$} be a ring homomorphism.
\begin{compactitem}
\item If \e{$S$} is a subring (ideal) in \e{$R$}, then \e{$\varphi (S)$} is a subring (ideal) in \e{$Im(\varphi )$}.
\item If \e{$S'$} is a subring (ideal) in \e{$Im(\varphi )$}, then \e{$\varphi^{-1} (S')$} is a subring (ideal) in \e{$R$}.
\end{compactitem}
\end{Le}
%\begin{PP} page 11 \end{PP}
Differn't subrings in the domain can have the same image in the codomain. If we restrict to only those subrings in the domain that contain the kernal, then we get a correspondance.
\begin{T}[Correspondance theorem] Let \e{$\varphi : R\to R'$} be a ring homomorphism. The maps 
$$\e{ \Phi : \{S\leq R: ker(\varphi) \subseteq S\} \mapsto \{S'\leq R' :S' \subseteq Im(\varphi) \}, \Phi(S) = \varphi(S)}$$
$$\e{ \Psi : \{I\triangleleft R: ker(\varphi) \subseteq I\} \mapsto \{I'\subseteq R' :I' \triangleleft Im(\varphi) \}, \Psi(I) = \varphi(I)}$$
are inclusion preserving bijections.\end{T}
\begin{R} The \dw{correspondence theorem}, states that if  \e{$N$} is a normal subgroup of a group \e{$G$}, then there exists a bijection from the set of all subgroups \e{$A\subset G$} s.t. \e{$N\subset A$}, onto the set of all subgroups of the quotient group \e{$G/N$}. The structure of the subgroups of  is exactly the same as the structure of the subgroups of  containing  with  collapsed to the identity element. \end{R}

\subsubsection{Notes from questions}
\begin{D} The \dw{sum} of two rings \e{$I,J$} is denoted \e{$I+J = \{x+y|x\in I,y\in J\}$} \end{D}
\begin{D} The \dw{characteristic} of a unital ring \e{$R$} is the smallest \e{$n\in\mathbb{N}$} such that \e{$1+1+\cdots +1=0$} (n times). \end{D}

\begin{R} The characteristic is the natural number \e{$n$$} such that \e{$R$} contains a subring isomorphic to the factor ring \e{$Z/nZ$}, which would be the image of that homomorphism.\end{R}

\begin{R} The characteristic of a field is either zero or a prime number \end{R}
\begin{R} If \e{$R$}, \e{$S$} are two rings, and there exists a homomorphisms betweeen them, then the characteritic of \e{$S$} divides the characteristic of \e{$R$}. \end{R}

\begin{Le} Every integral domain either contains a subring isomorphic to \e{$\mathbb{Z}$} or contains a subring ismorphic to \e{$\mathbb{F}$}.
\begin{D} A \dw{prime field} is a field with no proper subfields. \end{D}
\begin{Le} A prime field is isomorphic to either \e{$\mathbb{Q}$} or \e{$\mathbb{F}_p$} for some prime \e{$p$} \end{Le}
\begin{D} The \dw{Field of quotients} of an integral domain \e{$D$} is  \e{$F=\{(a,b)|a,b\in D,b\not= 0\}$} where \e{$(a,b)\sim (c,d), if, ad=bc$} and operations given by 
$$\e{(a,b)+(c,d) = (ad+bc,bd)}$$
$$\e{(a,b)(c,d)-(ac,bd)}$$
\end{D}











\subsection{4: Constructions}
\begin{Le} The \dw{direct product} \e{$R\times S$} of two rings is a ring given by the set \e{$\{(r,s)|r\in R, s\in S\}$} with operations given by 
$$\e{(r_1,s_2)+(r_2,s_2)=(r_1+_R r_2,s_1+_S s_2)}$$
$$\e{(r_1,s_2)\times (r_2,s_2)=(r_1\times_R r_2,s_1\times_S s_2)}$$ where the operations \e{$+,\times $} are inherited from \e{$R,S$} respectivly. \end{Le} The operation of taking diret products is (up to isomorphism) associative and commutative: \e{$ R\times (S\times T) \cong (R\times S)\times T$} and \e{$R\times S \cong S\times T$}.

\begin{D} Let \e{$R$} be a commutative unital ring. The \dw{polynomial ring} of has elements of the form \e{$\sum_{i=0}^n a_iX^i$}. The ring \e{$R$} embeds in \e{$R[X]$} as the degree zero polynomials. If \e{$R$} is unital/commutative/ID then so is \e{$R[X]$}. \end{D}

\begin{D} A matrix ring is invertible iff it's determinant is a unit \end{D}
\begin{D} The ring of \dw{endomorphisms} is the set of all ring homomorphisms from \e{R} to itself. The operations are pointwise addition and composition. 
$$\e{(f+g)(a)=f(a)+g(a)}$$
$$\e{(fg)(a) = f(g(a))}$$
\end{D}
\begin{D} Let \e{$G$} be a group and \e{$R$} be a commutative untial ring. The \dw{goup ring} is the set 
$$\e{R(G) = \{\sum_{i=0}^n a_ig_i, n\in \mathbb{N}^+, a_i\in R, g_i\in G\}}$$
\e{$R(G)$} forms a ring. \end{D}

\subsection{5: Generating sets and principle ideals}
\begin{D} Let \e{$R$} be a ring and \e{$X\subseteq R$} be a subset. The \dw{subring generated by} \e{$X$} is is the intersection of all subrings of \e{$R$} that contain \e{$X$}.\end{D}
\begin{D} Let \e{$R$} be a ring and \e{$X\subseteq R$} be a subset. The \dw{ideal generated by} \e{$X$} is the intersection of all ideals in \e{$R$} that contain \e{$X$}. Say \e{$X=\{a_1,\cdots,a_n\}$}, then the \dw{ideal generated} by \e{$X$} is denoted \e{$(a_1,\cdots,a_n)$}.\end{D}


\begin{D} An ideal \e{$I\triangleleft R$} satisfying \e{$I=(a)$} for some \e{$a\in R$} is known as a \dw{principle ideal}. \end{D}

\begin{Le} Let \e{$R$} be a commutative ring and \e{$X\subseteq R$}, then, 
\e{$(X) = \{\sum_1^n r_ia_i : n \in \mathbb{N}^*, r_i\in R, a_i\in X \}$}. That is the ideal generated by \e{$X$} is the set of all \e{$R$}-linear combinations of the elements of \e{$X$}.\end{Le}

\begin{D} A \dw{principle ideal domain} is an integral domain on which all ideals are principle \end{D} 

\begin{R} Any field \e{$F$} is trivially a PID as \e{$F$} contains only two ideals \e{$\{0\},F$} \end{R}
\begin{R} \e{$\mathbb{Z}[X]$} is not a PID as \e{$(2,X)$} is not principle. \end{R}


\subsection{6: Division and factorization in integral domains}
In this section \e{$R$} is always an \dw{integral domain}.

\begin{D} We say that \e{$a$} \dw{divides} \e{$b$} if \e{$\exists c \in R$} s.t. \e{$ b=ac$} and denote \e{$a|b$} \end{D} 
\begin{R} Think of the \e{$|$} sign as \e{$constant=$}. \end{R}

\begin{D} We say that \e{$a$} and \e{$b$} are \dw{associates} if both \e{$a|b$} and \e{$b|a$} and denote associates as \e{$a\sim b$} \end{D}

\begin{R} \e{$a|1$} iff \e{$a$} a unit. \end{R}
\begin{R} \e{$a|b, \forall b$} if a is a unit \end{R}
\begin{R} \e{$a|b$} iff \e{$(b)\subseteq (a)$} \end{R}
\begin{R} \e{$a\sim b \Longleftrightarrow (a)=(b) \Longleftrightarrow a=bu$} for some unit \e{$u$} \end{R}

\begin{Le} \e{$a|b$} is equivolant to \e{$b\in (a)$} \end{Le}

\begin{D} An element \e{$a\in R$} is called \dw{irreducible} is \e{$a$} is not a unit and the following holds $$\e{a=bc \implies b \text{ is a unit or }c\text{ is a unit }}$$ 
That is: all divisors of \e{$a$} are either units or associates of \e{$a$}.\end{D}

\begin{D} An element \e{$a \in R\backslash \{0\}$} is called \dw{prime} is \e{$a$} is not a unit and the following holds
$$\e{a|bc \implies a|b\ or \ a|c}$$
\end{D}

\begin{T} In an integral domain, prime elements are irriducible \end{T}

\begin{D} An integral domain \e{$R$} is called a \dw{unique factorization domain} if 
\begin{compactitem}
\item \dw{Existance of factorization}: every element \e{$a\in R$} that is non-zero and not a unit can be written as a product of irriducibles \e{$a=a_1a_2\cdots a_n$}.
\item \dw{Uniquenes of factorization}: If there is another factorization of \e{$a$} into a product of irriducibles then the two factorizations differ only by re-ordering and replacing each factor by an associate. 
\end{compactitem}
\end{D}

\begin{Le} In a UFD all irriducible elements are prime \end{Le}


\subsection{7: Prime and maximal ideals}
\begin{D} Let \e{$R$} be a commutative ring, and \e{$I\not=R$} an ideal in \e{$R$}. 
Then \e{$I$} is said to be \dw{prime} if it satisfies the condition: \e{$\forall a,b \in R, ab\in I \implies a\in I, or, b\in I$}\end{D}

\begin{D} Let \e{$R$} be a commutative ring, and \e{$I\not=R$} an ideal in \e{$R$}. 
Then \e{$I$} is said to be \dw{maximal} if it satisfies the condition: \e{$\forall J\triangleleft R
, I\subseteq J \implies J=I or J=R$}
\end{D}


\begin{T} Let \e{$R$} be a commutative unital ring, and \e{$I\triangleleft R$} an ideal in \e{$R$}. Then
\begin{compactitem}
\item \e{$I$} is prime \e{$\Longleftrightarrow$} \e{$R/I$} is an integral domain
\item \e{$I$} is maximal \e{$\Longleftrightarrow$} \e{$R/I$} is a field
\end{compactitem}
\end{T}

\begin{C} Every maximal ideal is prime \end{C}

\begin{Le} Let \e{$R$} be an integral domain, and \e{$a\in R\backslash \{0\}$} then 
\begin{compactitem}
\item If the ideal \e{$(a)$} is maximal, then \e{$a$} is iriducible
\item Suppose that \e{$R$} is a PID. If \e{$a$} is irriducible, then \e{$(a)\triangleleft R$} is maximal.
\end{compactitem}
\end{Le}

\begin{Le}
Let \e{$R$} be an integral domain and \e{$a\in R\backslash \{0\}$}. The ideal \e{$(a)\triangleleft R$} is prime iff \e{$a$} is a prime element
\end{Le}

\subsection{8: \e{$\mathbb{R}[X]$} is a PID}
\begin{T} Given \e{$f,g\in F[X]$} with \e{$g\not=0$}, there exist polynomials \e{$q,r\in F[X]$} s.t. \e{$f=qg+r$} and either \e{$deg(r)<deg(g)$} or \e{$r=0$}. Moreover the polynomials \e{$q$} and \e{$r$} are unique.\end{T}

%\begin{T} Take \e{$fg\in F[X]$} where \e{$F$} is a field, with \e{$g\not= 0, \exists qr\in F[X]$} s.t. \e{$f=qg+r$} where \e{$deg(r ) < deg(g )$} or \e{$r=0$}. Moreover the polynomials \e{$q,r$} are unique \end{T}

\begin{T} The condition that \e{$F$} must be a field can be relaxed to \e{$F$} just being an integral domain and that the leading coeffiecant of \e{$g$} is a unit \end{T}

\begin{C} Let \e{$f\in F[X]$}. Then \e{$a\in F$} is a \dw{root} of \e{$f$} iff \e{$(X-a)|f$} \end{C}

\begin{T} Let \e{$F$} be a field. Then \e{$F[X]$} is a PID. The converse is also true: Let \e{$F[X]$} be a PID, then \e{$F$} is a field. \end{T}

\subsection{9 Every PID is a UFD}

\begin{D} Let \e{$R$} be a commutative ring. Then we say that \e{$R$} satisfies the \dw{ascending chain condition} (ACC) if for all chain of ideals in \e{$R$}
$$\e{I_1\subseteq I_2 \subseteq \cdots \subseteq I_n \subseteq \cdots, \exists n\in \mathbb{N} \text{s.t.} I_i=I_n \forall i\geq n}$$
\end{D}

\begin{Le} Let \e{$R$} be a commutative unital ring. If \e{$R$} satisfies the ascending chain condition then every non unit, non zero element or \e{$R$} can be written as a product of irriducibles. \end{Le}

\begin{Le} Let \e{$R$} be a PID. Then \e{$R$} satisfies the ascending chain condition. \end{Le}

\begin{Le} Let \e{$R$} be a commutative unital ring in which all ideals are finitely generated iff \e{$R$} satisfies the ACC.\end{Le}

\begin{Le} Suppose that \e{$R$} is an integral domain in which all irreducible elements are prime. Suppose that \e{$a_1,a_2,\cdots, a_n,b_1,b_2,\cdots,b_m \in R$} are irreducible elements such that 
$$\e{a_1a_2\cdots a_n\sim b_1b_2\cdots b_m}$$
Then \e{$m=n$} and there is a permutation \e{$\pi$} of \e{$\{1,2,\cdots,n\}$}s.t. \e{$b_i\sim _{\pi (i)}$}
\end{Le}

\begin{Le} In a PID irreducible elements are prime \end{Le}

\begin{T} Every PID is a UFD \end{T}

\begin{C} For any field \e{$F$}, the polynomial ring \e{$F[X]$} is a UFD \end{C}

\subsection{ 10: \e{$\mathbb{R}[X_1,\cdots, X_n]$} is a UFD}
\begin{D}
Let \e{$R$} be a commutative ring. A \dw{greatest common divisor} of a finite number of elements \e{$a_1,\cdots a_n\in R$} is \e{$d\in R$} s.t.
\begin{compactitem}
\item \e{$d$} is a \dw{common divisor}: \e{$d|a_i, \forall i \in \{1,\cdots, n\}$} 
\item if \e{$d'$} is another common divisor, then \e{$d'|d$}. That is \e{$d\sim d'$}.
\end{compactitem}
\end{D}
 
\begin{Le} In a UFD a greatest common divisor of a finite number of elements (where at least one element is non zero) exists. \end{Le}

\begin{D} A collection of elements in a UFD is called \dw{relatively prime} iff it's gcd is a unit \end{D}

\begin{D} A polynomial \e{$\sum_{i=0}^n a_iX^i$} is called \dw{primitive} if it is non constant and the coefficients are relatively prime \end{D}

\begin{Le} Let \e{$R$} be a UFD and \e{$f\in F[X]$} be a non-constant polynomial. The  \e{$\exists a \in R$} and a primitive polynomial \e{$\hat{f} \in R[X]$} s.t. \e{$f=a\hat{f}$}. Moreover \e{$a,\hat{f}$} are unique up to associates. \end{Le}

\begin{Le} Let \e{$R$} be a UFD. If \e{$f,g\in R[X]$} are primitive so too is there product \e{$fg$} \end{Le}

\begin{Le} Let \e{$R$} be a field and \e{$F$} its field of quotients. Let \e{$f\in R[X],g_1,g_2\in F[X]$} be such that \e{$f=g_1g_2$}.  Then there exists \e{$g_1',g_2'\in R[X]$} with \e{$f=g_1'g_2'E$} and \e{$g_i \sim g_i'$}. Moreover, if \e{$g_i$} is in \e{$R[X]$} and is \dw{primitive}, we can take \e{$g_1'=g_1$}. \end{Le}

\begin{Le} If \e{$f\in R[X]$} is irriducible in \e{$R[X]$} and \e{$deg(f)\geq1$} then \e{$f$} is \dw{irriducible} in \e{$R[X]$}. \end{Le}

\begin{Le} Let \e{$f,g\in R[X]$} with \e{$deg(g)\geq 1$}. If \e{g|f$} in \e{$F[X]$} then \e{$g|f$} in \e{$R[X]$}. \end{Le}

\begin{Le} If \e{$f\in R[X]$} is primitive in \e{$R[X]$} and irriducible in \e{$F[X]$} then it is irriducible in \e{$R[X]$}. \end{Le}

\begin{P} Every non-zero, non-unit element in \e{$R[X]$} can be written as a product of irriducibles. \end{P}

\begin{P} Irriducible elements in \e{$R[X]$} are prime \end{P}

\begin{T} If \e{$R$} is a unique factorisation domain, then so too is \e{$R[X]$}. \end{T}

\subsection{11: Irriducible polynomials}
\begin{T} Let \e{$f=\sum_{i=1}^n a_i X^i \in \mathbb{Z}[X]$} with \e{$n\geq 1$}. Suppose there is a prime integer \e{$p\in\mathbb{Z}$} s.t.: 
\begin{compactitem}
\item \e{$p|a_i\forall i\in\{0,\cdots,n-1\}$}
\item \e{$p\not|a_n$}
\item \e{$p^2\not| a_0$}
\end{compactitem}
Then \e{$f$} is irriducible in \e{$\mathbb{Q}[X]$}.
\end{T}

\begin{C} Let \e{$p\in\mathbb{N}$} be prime. The polynomial \e{$ X^{p-1}+X^{p-2}+\cdots+X+1$} is irriducible in \e{$\mathbb{Q}[X]$}. \end{C}

\begin{P} Let \e{$f=sum_{i=0}^n a_iX^i\in \mathbb{Z}[X]$} with \e{$n\geq 1$} and \e{$p\in\mathbb{N}$} a prime that does not divide \e{$a_n$}.  If \e{$\bar{f}=sum_{i=0}^n \bar{a_i}X^i\in (\mathbb{Z}/(p))[X]$} is iriducible, then \e{$f$}i irriducible in \e{$\mathbb{Q}[X]$}. \end{P}

\begin{T} FACTORIXATION ALGORITHM IN \e{$\mathbb{Z}[X]$}. Given \e{$f\in\mathbb{Z}[X]\{0,1,-1\}$}
\begin{compactitem}
\item If \e{$deg(f)=0$}, then factorise in \e{$\mathbb{Z}$}. 
\item Otherwise let \e{$m=floor(\frac{n}{2})\in\mathbb{N},$} an calculate \e{$f(0),f(1),...,f(m)$}
\begin{compactitem}
\item If \e{$f(a) = 0$} for \e{$0\leq a\leq m$}, then \e{$(X-a)$} is a factor of \e{$f$}. If \e{$f=+-(X-a)$} then \e{$f$} is irriducible. If not \e{$f$} is reducible. Write \e{$f=(X-a)f'$} and start again. 
\item If \e{$f(a)\not= 0, \forall a\in \{0,\cdots,m\}$},let \e{$D=\{(d_0,d_1,\cdots,d_m)\in\mathbb{Z}^{m+1}|d_i|f(i)\}$}. This is a finite set. For each \e{$d=(d_0,d_1,\cdots,d_m)\in D$} let \e{$g_d\in\mathbb{Q}[X]$} be the unique polynomial with \e{$deg(g_d)\leq m$} and \e{$g(i)=d_i,\forall i\in\{0,1,\cdots,m\}$}. 
\begin{compactitem}
\item If there is a \e{$d\in D$} s.t. \e{$g_d$} is a proper factor of \e{$f$} in \e{$\mathbb{Z}[X]}$}, then we write \e{$f=g_df'$} and start again. 
\item If no \e{$g_d $} is a proper factor of \e{$f$} the \e{$f$} is irriducible.
\end{compactitem}
\end{compactitem}
\end{compactitem}
\end{T}

\subsection{12: Euclidean Domains}
\begin{D} A \dw{Euclidean Domain} is an integral domain \e{$R$} s.t. there exists a function \e{$\sigma:R\backslash \{0\}\rightarrow \mathhbb{N}$} satisfying 
\e{$$\forall a,b\in R, \text{ with }b\not= 0, \exists q,r,\in R \text{ s.t. } a=bq+r\text{ and either } r=0, or, \sigma(r)<\sigma(b)$$}
The function is called the \dw{norm function}. \end{D}
\begin{compactitem}
\item The definition does not reuire \e{$q,r$} to be unique.
\item There may be differn't maps \e{$\sigma$} that show that \e{$R$} is a Euclidean domain. Given such a \e{$\sigma$}, define \e{$\sigma':R\backslash \{0\}\rightarrow \mathbb{N}$} by \e{$\sigma'(a)....$}
\end{compactitem}

\begin{T} Every Euclidean domain is a PID \end{T} 

\begin{R} We have seen the following implications 
$$\e{ED\implies PID\implies UFD \implies ID}$$
\end{R}

\begin{A}
The Euclidean Algorithm. Let \e{$R$} be a \dw{ED} with norm function \e{$\sigma$}. Given two elements \e{$a,b\in R$} with \e{$b\not= 0$}, proceed as follows:
\begin{compactitem}
\item Let \e{$i=0,a_0=a,b_0=b$}
\item Write \e{$a_i=b_iq_i+r_i$} with \e{$r_i=0$} or \e{$\sigma(r_i)<\sigma(b_i)$}
\item If \e{$r_i=0$} then stop with answer \e{$b_i$}
\item Otherwise, let \e{$a_{i+1}=b_i$} and \e{$b_{i+1}=r_i$}
\item Increment \e{$i$} by one and go to step 1
\end{compactitem}








\section{Modules}
\subsection{Fundemental concepts}
A \dw{module} is a generalisation of a vector space: scalars do not nessecarily form a field, but may be any (commutative unital) ring. A module in which the scalars are a vector field is the same as a \dw{vector space}. A module in which the scalars are the integers is an \dw{abelian group}. 

\begin{D} Let \e{$R$} be a commutative unital ring. An \e{$R$}\dw{-Module} \e{$M$} is an abelian group whose operation we will denote by addition, together with a map \e{$R\times M \to M$} satisfying the following \e{$\forall \rho, \sigma \in R$} and all \e{$m,n\in M$}
\begin{compactitem}
\item \e{$1m=m$}
\item \e{$(\rho\sigma)m=\rho(\sigma m)$}
\item \e{$(\rho + \sigma)m = \rho m + \rho n$}
\item \e{$\rho(m+n) = \rho m+ \rho n$}
\end{compactitem}
We also call \e{$M$} a \dw{module over } \e{$R$}. The elements of the ring \e{$R$} and \e{$M$} will be reffered to as  \dw{scalars} and \dw{vectors} respectivly. We will sometimes denote a \e{$R$}-module by \e{$_RM$}. \end{D}

\begin{R} A ring \e{$R$} is an \e{$R$}-module. An ideal \e{$I\triangleleft R$} is an \e{$R$}-module. \end{R}
\begin{R} Any abelian group can be regarded as a \e{$\mathbb{Z}$}-module, and vice-versa, \end{R}

\begin{D} A \dw{submodule} of an \e{$R$}-module \e{$M$} is a subset that itself forms an \e{$R$}-module when using the operations inherited from \e{$M$}. \end{D}

\begin{Le} A subset \e{$N\subseteq M$} is a submodule iff 
\begin{compactitem}
\item \e{$N$} is non-empty
\item \e{$u,v\in N \implies u+v\in N$}
\item \e{$u\in N, \rho\in R \implies \rho u \in N$}
\end{compactitem}
\end{Le}

\begin{D} A \e{$R$}-module \dw{homomorphism} is a map \e{$\phi:V\to W$} between \e{$R$}-modules such that \e{$\forall u,v\in V, \rho\in R$}:
\begin{compactitem}
\item \e{$\varphi(u+v)=\varphi(u)+\varphi(v)$}
\item \e{$\varphi(\rho u)=\rho\varphi(v)$}
\end{compactitem}
\end{D}

\begin{D} A bijective homomorphism is a \dw{isomorphism} \end{D}

\begin{D} Given a submodule \e{$W$} of \e{$V$}, the \dw{quotient module} \e{$V/W$} is given by the aditive cosets \e{$\{v+W|v\in V\}$} with the operations:
\begin{compactitem}
\item \e{$(u+W)+(v+W)=(u+v)+W$}
\item \e{$\rho(u+W)=\rho u +W$}
\end{compactitem}
\end{D}

\subsubsection{From excercises}
\begin{T} Let \e{$\varphi : M \to N$} be an \e{$R$}-module homomorphism. Then \e{$Im(\varphi) \cong M/ker(\varphi)$} \end{T}

\begin{T} Let \e{$U$} and \e{$V$} be submodules of an \e{$R$}-module \e{$M$}. Then \e{$(U + V )/V \cong U/(U \cap V )$} \end{T}

\begin{T} Let \e{$U$} and \e{$V$} be submodules of an \e{$R$}-module \e{$M$} with \e{$U \subseteq V$} . Then \e{$(M/U)/(V/U) \cong M/V$} ?????????????????? changed M/U to M/V \end{T}

\begin{T} Let \e{$N$} be a submodule of an \e{$R$}-module \e{$M$}. The map \e{$S \to S/N$} sets up a one-to-one correspondence between the set of all submodules of \e{$M$} containing \e{$N$} and the set of all submodules of \e{$M/N$}.
The inverse of the map is \e{$T\to \pi^{-1}(T)$}, where \e{$\pi$} is the projection: \e{$M\to M/N, \pi(u)=u+N$}. \end{T}



\subsection{2 Free modules and bases}
\begin{D} Let \e{$S$} be a subset of a module \e{$M$}. The \dw{submodule generated} by \e{$S$} is the intersection of all submodules that contain \e{$S$}. This is easily seen to be a submodule of \e{$M$} and is denoted by \e{$\langle S \rangle$}. If \e{$\langle S \rangle = M$} we say that \e{$S$} is a \dw{generating set} for \e{$M$}. \end{D}

\begin{Le} \e{$\langle S \rangle = \{\sum_{i=0}^k \rho_iu_i | k\in\amthbb{N}, \rho_i\in R, u_i\in S$} \end{Le}

\begin{D} A subset \e{$S\subseteq M$} that is linearly independant if there exists \e{$\rho_1,\cdots,\rho_n\in R$} at least one  of which is non zero, and \e{$u_1,\cdots,u_n\in S$} such that \e{$\sum_{i=1}^n \rho_iu_i = 0$}. A subset that is not linearly dependant is called \dw{linearly independant} \end{D}

\begin{D} A subset \e{$S\subseteq M$} that is linearly independant and which is a generating set for \e{$M$} is called a \dw{basis} of \e{$M$}. If there is a basis for \e{$M$}, then \e{$M$} is called a \dw{free module}. \end{D}

\begin{R} All modules over a field are free \end{R}

\begin{Le} Let \e{$M$} be a \e{$R$}-module. A subset \e{$S\subset M$} is a basis of \e{$M$} iff every elemetn of \e{$M$} can be written uniquely as a linear combination of elements from \e{$S$}. \end{Le}

\begin{Le} Let \e{$M$} be a \e{$R$}-module and \e{$S\subseteq M$} a basis of \e{$M$}. THen any map \e{$S$} to an \e{$R$}-module \e{$N$} extends uniquely to a homomorphism from \e{$M$} to \e{$N$}. THat is, given a map \e{$f:S\to N$}, there is a unique \e{$R$}-module homomorphism \e{$\varphi:M \to N$} s.t. \e{$\varphi |_S = f$}.
\end{Le}

\begin{Le} If \e{$M$} is a free module with basis \e{$\{u_1,\cdots,u_n\}$} then \e{$M\cong R^n$}. \end{Le}

\begin{Pp} Suppose that \e{$R$} is an integral domain and \e{$m,n\in \mathbb{N}$}. Then \e{$R^m\cong R^n$} ( as \e{$R$}-modules) iff \e{$m=n$}. 
\end{Pp}

\begin{Pp} Every finitely generated \e{$R$}-module is a holomorphic image of a free \e{$R$}-module of finite rank \end{Pp}

\subsection{Torsion} 
\begin{D} The \dw{annihalator} of an eement \e{$u\in M$} in an \e{$R$} -module is \e{$ann_R(u) = \{\rho\in R|\rho u = 0\}$} \end{D}
\begin{D} An elemment \e{$u\in M$} is said to be \dw{torsion} if \e{$ann_R(u)\not= \{0\}$}. The \dw{torsion submodule} \e{$T_M$} consists of all torsion elements in \e{$M$}, that is,
$$\e{T_M=\{u\in M|\exists \rho \in R\backslash \{0\},\rho u= 0\}}$$ \end{D}
\begin{D} The module \e{$M$} is said to be a \dw{torsion module} if all the elements in \e{$M$} are torsion, and \dw{torsion free} is zero is the only torsion element. \end{D}

\begin{Pp} Let \e{$M$} be a module over an integral domain \e{$R$}. The quotient module \e{$M/T_m$} is torsion free. \end{Pp}

\subsection{4 Submodules of free modules}
\begin{Le} Let \e{$R$} be a commutative unital ring. Let \e{$F$} be a free \e{$R$}-module and \e{$M$} an \e{$R$}-module. Let \e{$\varphi : M\to F$} be a surjective homomorphism. Then there exists a submodule \e{$F'\subseteq M$} s.t. \e{$F'\cong F$} and \e{$M=F'\oplus ker(\varphi )$} \end{Le}

\begin{T} Let \e{$R$} be a PID., and \e{$F$} a free \e{$R$}-module of finite rank \e{$r$}. Then every submodule of \e{$F$} is free and has rank at most \e{$r$}. \end{T}

\subsection{5 Matrices}
\begin{D}
Let \e{$R$} be an integral domain, and \e{$F$} and \e{$G$} two finitely generated free \e{$R$}-modules. Fix bases for \e{$\mathcal{B}=\{f_1,\cdots, f_nm\},\mathcal{C}=\{c_1,\cdots,c_n\}$} for \e{$F$} and \e{$G$}, every \e{$R$}-module homomorphism \e{$\varphi:G\to F$} is represented by a unique matrix \e{$M_{m\times n}(R)$} as follows. For each element \e{$g_j$} in the basis for \e{$G$} write \e{$\varphi(g_j)$} in terms of the basis for \e{$F$}, that is 
$$\e{\varphi(g_j)=\sum_{i=1}^ma_{ij}f_i}$$
The matrix \e{$(a_{ij})$} will be called the \dw{matrix of the homomorphism} \e{$\varphi$} with respect to the given bases, and will be denoted by \e{$[\varphi]_{\mathcal{B},\mathcal{C}}$} or simply \e{$[\varphi]$}. \end{D}

\begin{D} Two matrices \e{$A,B\in M_{m\times n}(R)$} are said to be \dw{equivolant} if there exist invertible matrices \e{$X\in M_{m\times m}$} and \e{$Y\in M_{n\times n}$} s.t. \e{$A=XBY$}. \end{D}
\begin{R} Equivolant matrices represent the same homomorphism, but with respect to differn't choices of bases \end{R}

\begin{Pp} Let \e{$R$} be a PID and \e{$A\in M_{m\times n}(R)$}. Then \e{$A$} is equivolant to a diagonal \e{$m\times n$} matrix \e{$diag(d_1,d_2,\cdots, d_{min(m,n)})$} satisfying \e{$d_1|d_2|\cdots |d_{min(m,n)}$}. \end{Pp}

\begin{R} Check page 46 for the algorithm to do so \end{R}

\subsection{6 Structure Theorem}
\begin{T} Let \e{$M$} be a finitely generated module over a principle ideal domain \e{$R$}. THen there exist elements \e{$d_1,d_2,\cdots,d_k \in R$} satisfying \e{$d_1|d_2|\cdots |d_k$} such that
$$\e{M\cong R/(d_1)\oplus R/(d_2)\oplus \cdots \oplus R/(d_k)}$$
\end{T}

\begin{C} Let \e[$M$} be a ifnitely generated module over  PID \end{C}
\begin{compactitem} 
\item If \e{$M$} is torsion free then \e{$M$} is free 
\item \e{$M=T_M\oplu F$}, where \e{$T_M$} is the torsion submodule of \e{$M$} and \e{$F$} is a free module of finite rank
\end{compactitem}

\begin{D} If any \e{$d_i$} is a unit, \e{$R/(d_i)\cong \{0\}$} so we can drop that summand from the decomposition of \e{$M$}. The decomposition \e{$(*)$} with all e\{$d_i$} non unit is called the invariant factor decomposition of \e{$M$}. THe non-unit elements \e{$d_i$} are called the \dw{invariant factors} of \e{$M$}. The non-zero non unit \e{$d_i$} are called the \dw{torsion invariants}. The number of \e{$d_i$} is called the \dw{torsion-free rank} of \e{$M$}. \end{D}

\begin{P} For a given \e{$M$} as in the Structure Theorem, the invariant factors are all uniquely defined by \e{$M$}, up to associates. the torsion-free rank is uniquely determined by \e{$M$}. \end{P}




















