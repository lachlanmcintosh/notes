\section{Probability and statistical Inference}
\begin{itemize}[QUESTIONS]
\item why cant we define a probability to the vents on slide 2 to be zero. 
\item how to show that \e{$\{T,H\}^{\mathcal{N}}$} on slide 5. Also on slide 5: what cna you use a countbale space of outcomes on?
\item are the things on slide 7 supposed to have the : for the dfinition of set operations?

\begin{D} A \dw{random experiment} has "mass character" (can be repeated many many times), "deterministic irregularity" (we cannot predict the outcome of the experiment from I.C.'s), and has "statistical regularity" (relative frequencies stabilize) \end{D}

\begin{D} THe \dw{sample space} \e{$\Omega$} of a random experiment is the set of all possible outcomes. \end{D}
\begin{D} The \dw{outcomes} are elements \e{$\omega \in Omega$} \end{D}
\begin{D} The \dw{events} are subsets \e{$A \subset \Omega$} on which probability is defined. \end{D}

\begin{D}[Product Space] The \dw{cartesian product space} is defined as \e{$A\times B := \{(a,b):a\in A, b\in B\}$} and furthermore \e{$A^n := A\times \cdots \times A = \{(a_1,\cdots,a_n):a_j\in A, j=1,2,...,n\}$}. $|

\begin{D}[Composite experiment] Take a RE with a sample space\e{$\Omega_0$} which is replicated \e{$n$} times. The compositie experiment is \e{$\Omega = \Omega_0^n$}, where \e{$\omega=(\omega_0,\cdots,\omega_n$}, s.t. \e{$ \omega_j \in \Omega_j$} is the outcome of the jth replication of the experiment. \end{D}

begin{D}[Countable sets] Any set \e{$A$} s.t. \e{$\exists$} a one to one mapping \e{$f:A\mapsto \mathcal{N}$} is said to be countable (denumerable)\end{D}

Probabillities are assigned to sets of outcomes (subsets of \${Omega$}: but not nessecarily all all subsets of \e{$\Omega$}. \dw{You can only assign probabiliites to all subsets of \e{$\Omega$} iff \e{$\Omega$} is countable!}.

Consider this: What is the probabiliity of picking a real number from [0,1]: intuition says zero, but try to sum all those probabilities and you do not get one. so it is impossible to assign probabilities. 

\begin{D}[Set Operations] Some basic set operations frequently used in probabillity are: begin{itemize}
\item \e{$A^c := \{\omega\in Omega : \omega \notin A\}$}, the \dw{complement}
\item \e{$B\cap C := \{\omega\in Omega : \omega \in B, \omega \in C\}$}, the \dw{intersection}
\item \e{$D\cup E := \{\omega\in Omega : \omega \in D OR \omega \in E\}$}. the \dw{union}
\item \e{$F\G := F\cap G^c$}: the \dw{set difference}: everything in F and not in G.
\item \e{$A\triangle B:= A\B+B\A \equiv (A\cap B^c)\(A^c \cap B)$} is the symmetric set difference of the two sets \e{$A$} and \e[$B$}. 
\item \e{$A^c \cap B^c = (A\cup B)^c$}; are \dw{demorgans laws)
\end{itemize}
\end{D}

\begin{D} We say that for two sets \e{$A,B \in \Omega$}, if \e{$A\cap B = \nullset$} then the events are \dw{disjoint} \end{D}
\begin{R} They are not joined so how can they overlap \end{R}

It can be easier to use functions that set operations. Thus the corresponding set operations are:
\begin{itemize} 
\item \e{$ 1_A(\omega) := \twopartdef{1,\omega\in A, 0, \omega \notin A}$},
\item \e{$1_{A^c} = 1 - 1_A$},
\item \e{$1_{A\cap B} = 1_A1_B$},
\item \e{$1_{A\cap B} = max{1_A,1_B}$}
\item \e{$1_{A\triangle B} = |1_A-1_B|$}
\end{itemize}

\begin{D} We are allowed to take countable unions (and intersections). In mathematcs when countable infinity is allowed/involved, one often uses \e{$\sigma$} to indicate that
\begin{D} A family \e{$\amthcal{F}$} os subsets of \e{$\Omega$} is said to be a \e{$\sogma$}-algebra on \e{$\Omega$} if \begin{itemize}
\item  \e{$\Omega \in \mathcal{F}$}
\item \e{$A \in \mathcal{F} \implies A^c \in \mathcal{F}$}
\item \e{$A_1,A_2,... \in \mathcal{F} \implies \bigcup_{n=1}^\infty A_n \in \mathcal{F} $}
\end{itemize}
In words: the family is
closed under complementation and countable union and
intersection (follows from De morgans laws). 
\end{D}

\begin{D} One starts modelling an RE by specifying a suitable sample space and then
choosing an appropriate \e{$\sigma$}-algebra \e{$\mathcal{F}$}
of subsets. The elements of this \e{$\sigma$} -algebra are called
\dw{events}. \end{D}

\begin{D}[Algebra of sets] If we only had a finite union (intersection) of events for axiom 3, we would have a \dw{algebra of sets} rather than a sigma algebra. \end{D}

\begin{D} The smallest $\sigma$-algebra is the \dw{$\sigma$-algebra generated by A: 
$$\sigma(A) := \{\emptyset ,A,A^c, \Omega \}$$ \end{D}

\begin{D} Let \e{$\mathcal{G} = \{ A-1,\cdots, A-n\}$} be a finite partition of \e{$\Omega$}. Then the \e{$\sigma$} - algebra generated by \e{$\mathcal{G}$} is the smallest \e{$\sigma$}-algebra that contains all the sets \e{$A_j$}, is 
$$ \sigma(\mathcal{G}) := \{\sum_{i\in I} A_i : I \subset \{1,2,...,n\}\} $$
\end{D}

\begin{T} For any family $\mathcal{G}$ of subsets $\Omega$, there exists a unique $\sigma$-algebra, denoted by $\sigma(\mathcal{G}) and called the \sigma - algebra generated by \mathcal{G}$, s.t
\begin{itemize}
\item \e{$\mathcal{G} \subset \sigma(\mathcal{G})$}, and
\item if \e{$\mathcal{H}$} is a $\sigma$-algebra and $\mathcal{G}\subset\mathcal{H}$}, then \e{$\sigma(\mathcal{G}) \subset \mathcal{H}$}
\end{itemize}
That is,\e{$\sigma{\mathcal{G}}$} is the smallest \e{$\sigma$}-agebra on \e{$\Omega$} containing \e{$\mathcal{G}$}

\begin{D} The \dw{Borel subsets} of \e{$\mathcal{R}$} is the generated $sigma$ algebra generated by \e{$\mathcal{R}$}:
$$\e{\mathcal{B}(\mathbb{R}^m):=\sigma\{\prod_{i=1}^m (a_i,b_i]:a_i,b_i\in\mathbb{R} , a_i<b_i\} }$$
\end{D}
\begin{R} ust remember they are of the form open-closed intervals \end{R}
When \e{$\Omega=\mathcal{R}^m, \mathcal{B}(\mathbb{R}^m)$} is the default choice of \e{$\mathcal{F}$}.
For \e{$\Omega \subset \mathbb{R}^m$}, one takes \e{$\mathcal{F} =\{\Omega\cap A: A\subset\mathcal{B}(\mathbb{R})\}$} call the \dw{trace} of \e{$\mathcal{B}(\mathbb{R}^m)$} on \e{$\Omega$}.

\begin{D}
Let \e{$(\Omega,\mathcal{F})$} be a sample space endowed witha $\sigma$-algebra of its subsets (th couple is called a \dw{measureable space} ). A \dw{probability} on \e{$(\Omega,\mathcal{F})$}  is a function \e{$\mathbb{P} : \mathcal{F} \mapsto \mathbb{R}$} s.t.
\begin{itemize}
\item \e{$\mathbb{P}(A) \geq 0, A\in \mathcal{F}$},
\item \e{$\mathbb{P}(\Omega)  = 1$},
\item for all pairwise disjoint events \e{$A_1,A_2,\cdots \in \mathcal{F}$},
$$\e{\mathbb{P}(\bigcup_{j=1}^\infty A_j) = \sum_{i=1} \mathbb{P}(A_j)}$$
This is known as \dw{countable additivity}\end{itemize}
\end{D}

\begin{D} The triple \e{$\Omega, \mathcal{F}, \mathbb{P}$} is defined as a \dw{Probability space} \end{D}
page 17