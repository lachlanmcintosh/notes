\section{Knowledge Technologies}
\subsection{Introduction}
%The purpose of the subject is to describe and apply fundamentals of knowledge systems, including data acquisition and aggregation, knowledge extraction, text retrieval, machine learning and data mining.
%\dw{Knowledge technologies} refers to the way we can use computation to enhance our knowledge: stepping beyond the calculation of concrete tasks. Knowledge tasks are tasks dealing with irregular or unreliable data: for when the outcome is not well defined to enable us to mae decisions or inferences.
%Computers do not nessecarily make decisions for us: rather they can medieate between us and data.
%The data may be created for the task, or might be derived from the physical world – transformed, by a device, into bits from entities or events in our universe.
%Consider effectiveness rather than correctness. (Can a document ranking possibly be “correct”?)
%SOME GOOD TEXTS:

\lstset{language=Java}

\begin{compactitem}
\item \dw{Rao Kotagiri}, Doug McDonelll Buiding Room 7.10
kotagiri@unimelb.edu.au
\item \dw{consultations}: 10:00am–11:00am Monday, 9.00-10.00 Wed
\item \dw{data} for the subject is kept in the directory \e{/home/subjects/comp90049/local/} on the CSS servers.
\item Christopher D. Manning, Prabhakar Raghavan, and Hinrich Schutze
(2008), Information Retrieval, Cambridge University Press.
Freely available at informationretrieval.org
\item Pang-Ning Tan, Michael Steinbach and Vipin Kumar (2005)
Introduction to Data Mining, Addison-Wesley.
\end{compactitem}

\subsection{String Search}

\dw{Regular expressions} (regex, regexp) are patterns that match character
strings. They can be thought of as describing a set of strings.
\begin{compactitem}
\item \dw{Search}: Find the strings in a file that contain a substring that
matches a given pattern (grep family).
\begin{lstlisting}
> egrep 'rudd' *.txt
> egrep 'col(o|ou)r' *.txt
\end{lstlisting}
\item \dw{Find and replace}: Substitute some new string for the matching
substring in vi.
\begin{lstlisting}
s/rudd/gillard/g
s/[dD]og/Canis lupis familiaris/g
\end{lstlisting}
\item \dw{Validate or test}: Check if new string is correct (awk, Python, Perl).
\begin{lstlisting}
$input ~ /gillard/
$input ~ /^[A-Z0-9._%+-]+@[A-Z0-9.-]+\.[A-Z]{2,4}$/
\end{lstlisting}
Here the operator \e{$\sim$} in the expression \e{input $\sim$ /string/},is
a match operator that checks whether input contains the pattern
\e{/string/}.
\end{compactitem}

The four main concepts of regex mirror the four types of structure in
imperative programming languages.\begin{compactitem}
\item \dw{Sequence}: i = 2; j = 3; 
\item \dw{Matching}: /cat/
\item \dw{Assignment}: i = 2; 
\item \dw{Memoization}: (pattern)
\item \dw{Selection}: if A: do thing else: do other thing; 
\item \dw{Alternation}: /cat|dog/
\item \dw{Repetition}: while True: i += 1; 
\item \dw{Repetition}: /(cat)*/
\end{compactitem}

As the examples above show, regular expressions are a mix of literal
characters and command or control characters. For example,
\begin{compactitem}
\item \e{a} means “match the character a”. 
\item \e{$|$} means or
\item \e{$ \{\{ \} [ ] ( )\ \textasciicircum{}\ \$ . | * + ? \} $} are known as \dw{metacharacters} and
need to be escaped by a backslash (\e{$\backslash$}) to be used in a literal match. Beware, some tools have different metacharacters. And in some cases \e{$\backslash$} turns a character into a metacharacter.
Here, we sometimes use \e{$/$} as a \dw{pattern delimiter}. In some tools, it too is
a metacharacter.
\item The \dw{wildcard} \e{.} is the most basic metacharacter. It matches any single character (except a newline):
\item The \dw{anchors} \e{\textasciicircum{}} and \e{\$} match the start and end of a line or string,
respectively.
\item The \e{$|$} metacharacter expresses \dw{alternation} or \dw{disjunction}. 
\begin{compactitem}
\item \e{$/a|b|c/$} matches \e{“a”, “b”, or “c”}.
\item \e{$/cat|dog/$} matches \e{“cat”} or \e{“dog”}.
\end{compactitem}
\item The precise number of characters to match may be unknown; instead,
we specify a \dw{repetition construction}.
Some repetitions involve an arbitrary number: \begin{compactitem}
\item \e{*}: zero or more of the preceding element
\item \e{?}: zero or one of the preceding element
\item \e{+}: one or more of the preceding element
\end{compactitem}
These are \dw{greedy}: they match as many characters as they can. So \e{.*}
will always match a complete string and \e{a.*b} will pick up the last \e{“b”} in
the string.
Sometimes we care, but only approximately, about number. \begin{compactitem}
\item \e{\{n\}}: exactly n of the preceding element
\item \e{\{m,n\}}: between m and n (inclusive) of the preceding element
\item \e{\{n,\}}: n or more of the preceding element
\item \e{\{,m\}}: up to m of the preceding element
\end{compactitem}
\item We can match a set of characters known as \dw{character classes}: \e{[0-9],[a-z],[A-Z],[A-Za-z],[aeiou]}, etc.
Observe also that within \e{[]}, metacharacters may be used in their literal
meaning. 
\item A second use of the \e{\textasciicircum{}} metacharacter is to negate character classes.
\e{/[\textasciicircum{}A-Za-z]/} matches any non-alpha character.
\item Some character classes are used so frequently that they have names:
\begin{compactitem}
\item \begin{verbatim} [0-9] = [[:digit:]] = \d \end{verbatim}
\item \begin{verbatim} [a-zA-Z0-9_] = [[:word:]] = \w \end{verbatim}
\item \begin{verbatim} [\ \t\r\n\f] = [[:space:]] = \s \end{verbatim}
\item \begin{verbatim} [^0-9] = \D \end{verbatim}
\item \begin{verbatim} [^a-zA-Z0-9_] = \W \end{verbatim}
\item \begin{verbatim} [^\ \t\r\n\f] = \S \end{verbatim}
\end{compactitem}

\item Placing a pattern in parentheses leads to the match being stored as a
\dw{variable}.
The first stored pattern has the name \e{$\backslash 1$}, the \e{n}th is \e{$\backslash n$}. 
\end{compactitem}

\subsubsection{Pythonic regex}
The regex \e{"\"} clashes with the python use. TO overcome this you must write "\\" in place of "\". To avoid this, prefix thestring with the literal r"string".

\begin{lstlisting} 
*?, +?, ?? 
\end{lstlisting}
The \e{'*'}, \e{'+'}, and \e{'?'} qualifiers are all greedy; they match as much text as possible. Sometimes this behaviour isn’t desired; if the RE \e{$<.*>$} is matched against \e{$'<H1>title</H1>'$}, it will match the entire string, and not just \e{$'<H1>'$}. Adding \e{$'?'$} after the qualifier makes it perform the match in \dw{non-greedy} or \dw{minimal fashion}; as few characters as possible will be matched. Using \e{$.*?$} in the previous expression will match only \e{$'<H1>'$}.

Likewise \e{$\{m,n\}?$} attempts to match the previous string with as few repitiions as possible.

\begin{lstlisting}
(...)
\end{lstlisting}
Matches whatever regular expression is inside the parentheses, and indicates the start and end of a group; the contents of a group can be retrieved after a match has been performed, and can be matched later in the string with the "\\number" special sequence


\begin{lstlisting}
(?P<name>...)
\end{lstlisting}
Similar to regular parentheses, but the substring matched by the group is accessible within the rest of the regular expression via the symbolic group name name. 

\begin{lstlisting}
(?P=name)
\end{lstlisting}
Matches whatever text was matched by the earlier group named name.

\begin{lstlisting}
(?#...)
\end{lstlisting}
A comment; the contents of the parentheses are simply ignored.
\begin{lstlisting}
(?=...)
\end{lstlisting}
Matches if ... matches next, but doesn’t consume any of the string. This is called a lookahead assertion. For example, Isaac (?=Asimov) will match 'Isaac ' only if it’s followed by 'Asimov'.


\begin{lstlisting}
(?!...)
\end{lstlisting}
Matches if ... doesn’t match next. This is a negative lookahead assertion. For example, Isaac (?!Asimov) will match 'Isaac ' only if it’s not followed by 'Asimov'.

\begin{lstlisting} 
(?<=...) and (?<!...)
\end{lstlisting}
Do the same as above but in the oposite direction.


\begin{compactitem}
\item \e{$\backslash A$} matches only at the start of the string.
\item \e{$\backslash b$} matches the empty string, but only at the beginning or end of a word.
\end{compactitem}

There are more special caracters here http://docs.python.org/2/library/re.html 

\begin{compactitem}
\item re.compile(pattern, flags=0)
Compile a regular expression pattern into a regular expression object
\begin{lstlisting}
prog = re.compile(pattern)
result = prog.match(string)
\end{lstlisting}
is equivalent to
\begin{lstlisting}
result = re.match(pattern, string)
\end{lstlisting}
finsih at http://docs.python.org/2/library/re.html
\end{compactitem}

\subsection{Search Algorithms}

\begin{A}[NAIVE MATCHING]
Try to find the query \e{q} in the string \e{t}. 
\begin{compactitem}
\item Begining witht the first letter in \e{q}, move along \e{t} until you have a match.
\item move to the next letter of \e{q}: if the next letter in \e{t} is not a match, go back to step one and move the first letter of \e{q} along again until another match is found.
\end{compactitem}
This method requires up to \e{$|q-1|$} comparisons for each of the \e{$|t|$} characters: thus complexity is \e{$|q-1||t|+1$}.
\end{A}

\begin{A}[Boyer-Moore matching]:
\begin{compactitem}
\item line the starts of the two words together. 
\item compare the end of the strings, if there is a mismatch take the search string mismatch letter and see if it is in our query, if so, align the query forward to that, otherwise, skip the length of the query.
\end{compactitem}
\end{A}

\subsection{Text Compression}
When a common string is recognized and replaced by a code, the
outcome is a \dw{codebook} and hopefully more compact text.
A specific form of this “dictionary” approach is to look backwards for the
most recent occurrence of the same string. Variations of this approach
are the basis of the zip family of compression algorithms: the \dw{LZ family}.

\begin{A}
A simple take on \dw{LZ compression}: \begin{compactitem}
\item Pretend that the alphabet is written to the left of the string.
\item Encode each substring in the string by a pointer to the left to an
occurrence of the same string (not nessecarily in the alphabet).
\item Each pointer is a pair consisting of \e{d}: distance to the left and \e{l}:
number of characters to copy.
\end{compactitem}
\end{A}

Is it better to search for the longest
match, or the nearest one?
It can be formally shown that such problems cannot be optimally solved
through local or greedy methods, discovery of the best solution involves trying every combination
of possible substitutions. 

LZ methods avoid the combinatorics altogether by using simple
heuristics that are experimentally found to produce “good enough”
solutions on typical data.
\begin{A}
The main heuristic is to do no search at all and instead \dw{build a
dictionary of string substitutions} made so far.
\begin{compactitem}
\item The initial dictionary consists of all single letters.
\item Only dictionary entries can be matched. The longest match is
always taken.
\item When a substitution such as ang is made in tangled, the entry
angl is added to the dictionary.
\end{compactitem}
\end{A}

Another heuristic is to allow a form of \dw{forward matching}, giving
run-length encoding. For example,
dogcatcatcatcatcatcatcatdog
dogcat(3,18)dog.

%Consolidate your understanding of the regular expression
%metacharacters; some useful references:
%docs.python.org/dev/howto/regex.html
%java.sun.com/docs/books/tutorial/essential/regex/
% Some readings:
%en.wikipedia.org/wiki/Boyer-Moore_string_search_algorithm
%en.wikipedia.org/wiki/Lempel-Ziv-Welch

\subsection{Approximate Matching}
Pattern Mmtching is ineffiecant for approximate matching, due to combinatorial complexity. 

There is still no good way of doing context-related search.

\subsubsection{Stemming}
There are two kinds of dictionaries used in spelling correction.
\begin{compactitem}
\item \dw{Literal}: a set of strings that must be matched exactly.
\item \dw{Root+stem}: a set of root words, a set of suffixes, and a set of rules
for applying suffixes to roots.
\end{compactitem}
\dw{Stemming} is the process of stripping suffixes automatically. It allows a
dictionary to be generalized, but many root-suffix combinations are not
correct spellings.

\subsubsection{Neighbourhood search}
Given a query string that is not in the dictionary, the task is to find the
\dw{nearest neighbours}. One approach is to generate all neighbours, and see if they are present by enumerating all of the single-character deletions, insertions, and
replacements. This gives \e{$N_1 = n A\cdot (n+1)+(A-1)\cdot n$} nearest neighbours if we assume a distance of 1. If we assumed a distnce of 2, \e{$N_2 = N_1 + A(n+1)A(n+2) +A(n-1)A(n-2) + An^2 + \cdots $} which does not even include combinations of insertions and dletions, and is \e{$O(n^2)$}

\subsubsection{Combinatoric search}
Current practical techniques use variants of binary search, where:
\begin{compactitem}
\item Strings are sorted, that is, grouped by prefix.
\item Strings are inspected character-by-character until it is clear that
their distance from the query is greater than k.
\item Due to the sorting, groups of strings that share a prefix can be
considered simultaneously.
\item The distance limit \e{k} must be chosen, or repeated search is required.
\item Additional work is required for ranking.
\item Because there is no sorting of a dictionary that ensures that similar
strings are close together, it is still not obvious what the best solution to
this search problem is.
\end{compactitem}

\subsubsection{Edit distances}
Can be used to rate a match. The simplest edit distance is the number of insertions, deletions, and
substitutions needed to turn one string into another. Gives a natural way of thinking about a ranking.

Assume an array \e{F} of size \e{$|q|\times|t|$}, to be used to compute the global edit
distance between \e{q} and \e{t}.
\begin{lstlisting}
lq = strlen(q); lt = strlen(t);
for( i=0 ; i<=lq ; i++ ) F[i][0] = i;
for( j=0 ; j<=lt ; j++ ) F[0][j] = j;

for( i=1 ; i<=lq ; i++ )
    for( j=1 ; j<=lt ; j++ )
        F[i][j] = min3(
            F[i-1][j] + 1, % insertion
            F[i][j-1] + 1, % deletion
            % match/miss match
            F[i-1][j-1] + equal(q[i-1], t[j-1]));
\end{lstlisting}
The value \e{$F[lq][lt]$} is the minimum number of edits between between
\e{$q$} and \e{$t$}. 

The \dw{local edit distance} is similar.
\begin{lstlisting}
lq = strlen(q); lt = strlen(t);
for( i=0 ; i<=lq ; i++ ) F[i][0] = 0;
for( j=0 ; j<=lt ; j++ ) F[0][j] = 0;

for( i=1 ; i<=lq ; i++ )
    for( j=1 ; j<=lt ; j++ )
        F[i][j] = max3(
            F[i-1][j] - 1, % insertion
            F[i][j-1] - 1, % deletion
            % match/miss match
            F[i-1][j-1] + equal(q[i-1], t[j-1]));
\end{lstlisting}
Here, equal() returns +1 if equal, �1 otherwise. 
The value of F[i][j] with the highest value represents the best
alignment.

The alignment contributing to the maximum score can be reconstructed
by tracing-back the pointers stored during the calculation.

This process takes \e{$O(n^2)$}.

\subsubsection{N-grams}
Let \e{$G_n(s)$} be the substrings of length \e{$n$} of \e{$s$}. The \e{n}-gram distance of \e{t} and \e{s} is
\e{$|G_n(s)| + |G_n(t)| -� 2\times|j_Gn(s)\cap\ _Gn(t|$}

The smaller the \e{n}-gram distance, the closer the match. Cost is
approximately \e{$O(|s| + |t|)$}, compared to \e{$O(|s|\times |t|)$} for an edit distance.

\subsubsection{Phonetic matching}
\dw{Phonetic matching} techniques allow searching for a name or word when
the exact spelling is unknown.

The \dw{Soundex} sound-alike matching technique is well-known, widely
used, simple to implement.
Transforms a string into a 4-character code that represents its sound:
\begin{compactitem}
\item Replace all but the first letter by one of seven single-digit codes:
\e{$a e h i o u w y \to 0, b f p v \to 1,c g j k q s x z \to 2, d t \to 3,l \to 4, m n \to 5, r \to 6$}
\item Remove doubles, then remove 0s.
\item Truncate to four symbols.
\end{compactitem}
There are \e{$6994 =26(1 + 6 + 62 + 63) distinct Soundex codes.

\subsubsection{editex}
\dw{Zobel’s editex}: Like the standard edit distance, but instead of having a
fixed penalty for insertion, deletion, and substitution, have two penalties:
\begin{compactitem}
\item High for letters that are never similar, such as “d” and “m”.
\item Low for letters that can give rise to similar sound, such as “m”–“n”
and “c”–“k”. We have 10 groupings:
\e{1. a e i o u y 2. b p 3. c k q 4. d t 5. l r 6. f p v 7. s x z 8. c s z 9. m n 10. g j}.

Every name has a pronunciation, or, more precisely, can be represented
as a string of phonemes. Perhaps these can be used for matching in
some way.
Two problems:
\begin{compactitem}
\item Deciding what pronunciation corresponds to a given string.
\item Comparing pronunciations.
\end{compactitem}
Phoenetic tables cna be used for similarity.

\subsubsection{Effiecancy measurement}
Measurement of efficiency is straightforward.
But we also need to know whether the method is useful. Measurement
of effectiveness involves human assessment of the quality of the results.

\dw{Recall}: The proportion of the correct matches that are retrieved.

\dw{Average precision} is defined as: \e{$AP=\frac{1}{R}\sum_{i=1}^d (\frac{r_i}{i}\cdot \sum_{j=1}^i r_j )$} where \e{$r_i$} is \e{$1$} (respectively, \e{$0$}) if the \e{$i$}th item in a ranking is relevant (respectively, irrelevant), there are \e{$d$} items in the ranking, and there are
in total \e{$R$} known relevant items.

We use \dw{baselines} to establish whether any proposed method is doing
better than “dumb and simple” – “dumb” methods often work surprisingly
well.

\dw{Baseline} – naıve or straightforward method that we would expect a rich
(or principled, or . . . ) method to better.
\dw{Benchmark} – established “best practice” technique that we are pitching
our method against.
The word “baseline” is often used as an umbrella term for both
meanings.

\subsubsection{Readings}
en.wikipedia.org/wiki/Bitap_algorithm
en.wikipedia.org/wiki/Spell_checker
www.googleguide.com/spelling_corrections.html
“Learning a Spelling Error Model from Search Query Logs” (LMS).
“Rank-Biased Precision for Measurement of Retrieval Effectiveness”
(LMS), sections 2 & 3.
“Phonetic String Matching: Lessons from Information Retrieval” (LMS).
Manning et al., chapters 3 & 8.













j
